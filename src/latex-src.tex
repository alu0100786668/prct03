\documentclass[a4paper, 12pt]{article}
\usepackage [utf8]{inputenc}
\usepackage [spanish]{babel}
\usepackage {graphicx}
\begin{document}
\title {Titulo del articulo}
\author {Adrian Cruz Guerra\\
	    Técnicas Experimentales~\footnote{Universidad de la laguna}
	    }
\date {\today}
\maketitle
\begin{abstract}

  En \LaTeX {}~\cite{Lam:86} es sencillo escribir expresiones 
  matemáticas como $a=\sum_{i=1}^{10} {x_i}^{3}$
  y deben ser escritas entre dos símbolos \$.
  Los superíndices se obtienen con el símbolo \^{}, y
  los subíndices con el símbolo \_.
  Por ejemplo: $x^2 \times y {\alpha + \beta}$.
  También se pueden escribir fórmulas centradas:
  \[h^2=a^2 + b^2\]
\end {abstract}

\section {Primera sección}
Si simplemente se desea escribir texto normal en LaTeX,
sin complicadas f\'ormulas matem\'aicas o efectos especiales
como cambios de fuente, entonces simplemente tiene que escribir
en espa\~nol normalmente. \par
Si desea cambiar de párrafo ha de dejar una linea en blanco o bien
utilizar el comando $\\par$.
No es necesario preocuparse de la sandría de los párrafos:
todos los párrafos se sangrarán automáticamente con la excepción
del primer párrafo de una sección.


Se ha de distinguir entre la comilla simple 'izquierda'
y la comilla simple 'derecha' cuando se escribe en el ordenador.
En el caso de que se quiera utilizar comillas dobesse han de escribir
dos caracteres 'coomilla simple', esto es, 
''comillas dobles''.


También se ha de tener cuidado con los guiones: se utiliza un único

guión para la separación de sílabas, mientras que se utilizan 
tres guiones seguidos para producir un gión de los que se usan
como signo de puntuación --- como en esta oración.
\includegraphics[width=0.5\textwidth]{imagen1.eps}
\begin {thebibliography}{00}
  \bibitem{Lam:86}
    Lamport, Leslie.
    TLA in pictures
    \emph {IEEE Transactions on Software Engineering},
    21(9), 768-775.
\end {thebibliography}
\bigskip
\begin {tabular} {|l|c|c|}
\hline
  Nombre & Edad & Nota \\ \hline
  Pepe   &   24 &   10 \\ \hline
  Juan   &   19 &    8 \\ \hline
  Luis   &   21 &    9 \\ \hline
\end {tabular}
\end {document}